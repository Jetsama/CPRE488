% Options for packages loaded elsewhere
\PassOptionsToPackage{unicode}{hyperref}
\PassOptionsToPackage{hyphens}{url}
%
\documentclass[10pt,a4paper]{article}
\usepackage[left=25mm,right=25mm]{geometry}
\usepackage{amsmath}
\usepackage{amsfonts}
\usepackage{amssymb}

\author{}
\date{}



\usepackage{listings}
\usepackage{color}

\definecolor{dkgreen}{rgb}{0,0.6,0}
\definecolor{gray}{rgb}{0.5,0.5,0.5}
\definecolor{mauve}{rgb}{0.58,0,0.82}

\lstset{frame=tb,
  language=C,
  aboveskip=3mm,
  belowskip=3mm,
  showstringspaces=false,
  columns=flexible,
  basicstyle={\small\ttfamily},
  numbers=none,
  numberstyle=\tiny\color{gray},
  keywordstyle=\color{blue},
  commentstyle=\color{dkgreen},
  stringstyle=\color{mauve},
  breaklines=true,
  breakatwhitespace=true,
  tabsize=3
}

\usepackage{multicol}
\usepackage{graphicx}
\usepackage{epstopdf}

\epstopdfDeclareGraphicsRule{.gif}{png}{.png}{convert gif:#1 png:\OutputFile}
\AppendGraphicsExtensions{.gif}
\usepackage{chngcntr}
\counterwithin*{equation}{section}
\counterwithin*{equation}{subsection}
\usepackage{amsmath}

\usepackage{float} 
\usepackage{hyperref}
\usepackage{amsmath}
\let\oldsubsection\subsection
\renewcommand{\subsection}{%
    \setcounter{equation}{0}%
    \oldsubsection%
}
\begin{document}
\begin{flushleft}
\begin{LARGE}CPRE 488 Spring 2024: Embedded Systems Design\\
HW 1: Pulse Position Modulation \\
\end{LARGE}

Jonathan Hess
\\\href{https://github.com/Jetsama/CPRE488/tree/main/HW1}{GitHub Page}
\end{flushleft}
\author{}
\date{}




\section*{Task 1: Relearning VHDL}
\begin{enumerate}
    \item Read Sections 7-7.1 of the Free Range VHDL book\footnote{http://class.ece.iastate.edu/cpre488/resources/free\_range\_vhdl.pdf}.
    \begin{enumerate}
        \item Describe an unclear aspect and post a question on the “In-Class Discussion” topic on Blackboard. If everything was clear, describe Listing 7.1 line-by-line.
        
        
        
        \item In Section 7.4, complete exercises 1 and 2.
    \end{enumerate}
\end{enumerate}

\section*{Task 2: Pulse Position Modulation}
\begin{enumerate}
    \item Compare PWM and PPM.
    PWM uses 
    \item Draw a 6-channel PPM frame with a 20ms period, specified in milliseconds and clock cycles.
    \item Design a Mealy-type state machine for capturing PPM frames (bubble diagram).
    \item Design a Mealy-type state machine for generating PPM frames (bubble diagram).
\end{enumerate}

\section*{Task 3: HW/SW Interface Table}
Read through the entire MP-1 document and provide a table with columns for:
\begin{itemize}
    \item Register Name
    \item Relative Register Offset
    \item Short Register Description
\end{itemize}

\section*{Task 4: High-level Architecture}
Draw the high-level architecture of the MP-1 PPM interfacing system illustrating the location and interconnection of primary components.

\end{document}
